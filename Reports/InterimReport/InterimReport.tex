\documentclass[12pt]{article}

% Packages
\usepackage[utf8]{inputenc} % allow utf-8 input
\usepackage[T1]{fontenc}    % use 8-bit T1 fonts
\usepackage{geometry}       % to change the page dimensions
\usepackage{graphicx}       % support the \includegraphics command and options
\usepackage{amsmath}        % for better mathematical formulas
\usepackage{amsfonts}       % for mathematical fonts
\usepackage{amssymb}        % for mathematical symbols
\usepackage{hyperref}       % for hyperlinks
\usepackage{lipsum}         % for generating filler text
\usepackage{float}          % for better figure placement

% Page geometry
\geometry{a4paper, margin=1in}

\begin{document}

%Title Page
\begin{titlepage}
    \centering
    \vspace*{5cm}

    \Large
    \textbf{Swarm Robotics: Exploration and Mapping in Simulated Environments}

    \vspace{1cm}

    Charlie Anthony [candNo: 246537]\\
    Supervisor: Dr Chris Johnson

    \vfill

    \vspace{1cm}

    \small
    Interim report\\
    Computer Science and Artificial Intelligence BSc

    \includegraphics[width=0.3\linewidth]{sussex_logo.jpg} %Replace 'logo.jpg' with the path to your University of Sussex logo


    \small
    Department of Informatics and Engineering\\
    University of Sussex\\
    November 2023
\end{titlepage}

%Table of Contents
\tableofcontents
\newpage

%Introduction
\section{Introduction}
Swarms exist everywhere in life. Nearly all organisms exhibit some form of swarming behaviours within their
communities. Starlings display impressive organisational behaviour, positioning themselves with respect to the
movement of their neighbours. Humans show swarm behaviours when moving in crowds, for example, moving around sports
venues or exiting buildings in emergencies. No matter how hard you look, regardless if the context, swarms are
typically present.\\
These behaviours can also be artificially created in robotics. Within the realm of computing, parallelising
processes is breaking barrier after barrier - swarm robotics brings the same benefits. Being able to divide and
conquer a problem has the ability to reduce computational complexity by whole orders of magnitude. Therefore, it
would be wasteful not to properly dedicate the time which this discipline deserves.\\
My agents will be placed within close proximity inside a simulated environment and then allowed to explore and
combine their findings; ultimately creating a visualization map of its environment. The agents will need to both
navigate the environment and avoid collisions, whilst creating an internal representation of its surroundings. The
best-case scenario for the swarm I am developing is a fully decentralised system in simulation.\\
I will initially explore this problem by creating SLAM simulations, and then attempting to apply similar techniques
to a centralised system. These initial simulations will employ techniques such as particle filters, loop closures
and [insert something here] in order to create a base-line representation of the environment.

\section{Professional and Ethical Considerations}
My project maintains compliance towards all ethical considerations, as there is minimal external involvement from
humans. The majority of my project will be carried out in simulation, therefore no ethical approval is required. Should
my project progress to physically implementing agents, considerations such as safety around the robots, will be considered.
All tests will be carried out in an environment where people cannot be hit, therefore mitigating any trip hazards.\\
Research in this project is within the professional competence of myself, as it significantly relies upon knowledge
obtained from modules such as "Acquired Intelligence and Adaptive Behaviour" and "Fundamentals of Machine Learning." I
will further ensure all relevant gaps in knowledge are explored through reading extensively in the area and communicating
any areas of concern with my supervisor.

\section{Related Work}
\subsection{SLAM}
SLAM (Simultaneous Localisation and Mapping) is a technique used in robotics to create a map of an unknown environment.
It is an important area of research in robotics as it is heavily used in autonomous vehicles, drones and vacuum cleaners;
allowing agents to understand and navigate their environement effectively.\\
SLAM can be broken down into two sub-problems: localisation and mapping. Localisation is the process of determining the
location of a robot in its environment, whilst mapping is the process of constructing a map of the environment. The maps
are constructed using data collected from sensors, such as cameras and laser scanners. A lot of existing work in SLAM is
based on single robot applications, however, there is a growing interest in multi-agent SLAM. One of the greatest
challenges in SLAM is crossing the simulation to reality gap, as in the real world, sensor readings are noisy and
environments are dynamic, which increases the complexity of the problem.

\subsection{Graph-based SLAM}
It works by having a robot move around its environment, whilst taking measurements of its surroundings. These measurements
are usually received by a sensor, such as a camera or laser scanner. The robot then uses these measurements to create an
internal representation of the map, by combining the measurements with its understanding of the route it has taken. This
technique is used in many applications, such as autonomous vehicles, drones and vacuum cleaners.\\
\subsection{Particle Filters}
\subsection{Swarm}
\subsection{Random Walks}

\subsection{Requirements Analysis}
%table with requirements and their justification
\begin{table}[H]
    \centering
    \begin{tabular}{|p{0.2\linewidth}|p{0.7\linewidth}|}
        \hline
        \textbf{Requirement} & \textbf{Justification}\\
        \hline
        \textbf{R1} & \textbf{The system must be able to create a map of its environment}\\
        \hline
    \end{tabular}
    \caption{Requirements and their justification}\label{tab:table}
\end{table}


\section{Project Plan}
\begin{figure}[H]
    \centering
    \includegraphics[width=0.8\linewidth]{gantt_chart.png}
    \caption{Gantt chart showing the project plan}
    \label{fig:gantt_chart}
\end{figure}
The execution of my project will be split into various phases, where each phase will focus on an area of development. The
majority of the project will be software development, therefore a large portion time will be spent here. Figure
\ref{fig:gantt_chart} shows the project plan, where the grey bars represent the time spent at each phase.

\section{Methods and Preliminary Results}


\subsection{Supervisor Meetings}


\section{Appendices}


\end{document}


