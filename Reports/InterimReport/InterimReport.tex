\documentclass[12pt]{article}

% Packages
\usepackage[utf8]{inputenc} % allow utf-8 input
\usepackage[T1]{fontenc}    % use 8-bit T1 fonts
\usepackage{geometry}       % to change the page dimensions
\usepackage{graphicx}       % support the \includegraphics command and options
\usepackage{amsmath}        % for better mathematical formulas
\usepackage{amsfonts}       % for mathematical fonts
\usepackage{amssymb}        % for mathematical symbols
\usepackage{hyperref}       % for hyperlinks
\usepackage{lipsum}         % for generating filler text

% Page geometry
\geometry{a4paper, margin=1in}

\begin{document}

%Title Page
\begin{titlepage}
    \centering
    \vspace*{5cm}

    \Large
    \textbf{Swarm Robotics: Exploration and Mapping in Simulated Environments}

    \vspace{1cm}

    Charlie Anthony [candNo: 246537]\\
    Supervisor: Dr Chris Johnson

    \vfill

    \vspace{1cm}

    \small
    Interim report\\
    Computer Science and Artificial Intelligence BSc

    \includegraphics[width=0.3\linewidth]{sussex_logo.jpg} %Replace 'logo.jpg' with the path to your University of Sussex logo


    \small
    Department of Informatics and Engineering\\
    University of Sussex\\
    November 2023
\end{titlepage}

%Table of Contents
\tableofcontents
\newpage

%Introduction
\section{Introduction}
Swarms exist everywhere in life. Nearly all organisms exhibit some form of swarming behaviours within their
communities. Starlings display impressive organisational behaviour, positioning themselves with respect to the
movement of their neighbours. Humans show swarm behaviours when moving in crowds, for example, moving around sports
venues or exiting buildings in emergencies. No matter how hard you look, regardless if the context, swarms are
typically present.\\
These behaviours can also be artificially created in robotics. Within the realm of computing, parallelising
processes is breaking barrier after barrier - swarm robotics brings the same benefits. Being able to divide and
conquer a problem has the ability to reduce computational complexity by whole orders of magnitude. Therefore, it
would be wasteful not to properly dedicate the time which this discipline deserves.\\
My agents will be placed within close proximity inside a simulated environment and then allowed to explore and
combine their findings; ultimately creating a visualization map of its environment. The agents will need to both
navigate the environment and avoid collisions, whilst creating an internal representation of its surroundings. The
best-case scenario for the swarm I am developing is a fully decentralised system in simulation.\\
I will initially explore this problem by creating SLAM simulations, and then attempting to apply similar techniques
to a centralised system. These initial simulations will employ techniques such as particle filters, loop closures
and [insert something here] in order to create a base-line representation of the environment.

\section{Professional and Ethical Considerations}
\lipsum[1-2]

\section{Related Work}
\lipsum[3-4]

\subsection{Requirements Analysis}
\lipsum[5]

\section{Project Plan}
\lipsum[6-7]

\section{Methods and Preliminary Results}
\lipsum[8-9]

\subsection{Supervisor Meetings}
\lipsum[10]

\section{Appendices}
\lipsum[11-12]

% References
\begin{thebibliography}{99}

\bibitem{ref1}
Author, I. (Year). \textit{Title of the work}. Publisher.

\bibitem{ref2}
Author, I. (Year). \textit{Title of the article}. Journal, Volume(Issue), Pages.

\end{thebibliography}

\end{document}
